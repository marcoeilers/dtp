\documentclass{beamer}

\usepackage{listings}
\usepackage{amssymb, amstext, amsmath}

\usetheme[style=plain]{uu}

\title[uubeamer example presentation]{Verification Challenge: A simple Sudoku solver}
\subtitle{}

\author{Robert Schmidtke \and Marco Eilers}

\institute[Computing Science]{
Utrecht University\\
Faculty of Science\\
Department of Information and Computing Sciences}

%\usetheme[style=fancy,sidebar=false,showpagenr=false]{uu}
%\usetheme{Berlin}
%\usecolortheme{uubeamer}
\begin{document}

%%%%

\begin{frame}
  \maketitle
\end{frame}

%%%%
\begin{frame}
  \frametitle{Overview}
  \begin{itemize}
    \item Three approaches to solve Sudokus automatically
    \item Problems reasoning about the solver in Coq
    \item Some attempts in Agda
  \end{itemize}
\end{frame}


\begin{frame}[fragile]{First approach}
  Generate all possible choices and filter the correct results:
  \begin{lstlisting}
Definition correct (b:board) : bool :=
andb3
  (all nodups (rows b))
  (all nodups (cols b))
  (all nodups (boxes b)).

Definition sudoku_naive (b:board) : list board :=
filter correct (mcp (choices b)).
  \end{lstlisting}
  \(\Rightarrow 9^{40} = 147808829414345923316083210206383297601\) possibilities
\end{frame}

\begin{frame}[fragile]{Second approach}
  Prune the generated choices first:
  \begin{lstlisting}
Definition prune (choices:matrix_choices) :
  matrix_choices :=
prune_by boxes (prune_by cols (prune_by rows choices)).  

Definition sudoku_naive_2 (b:board) : list board :=
filter correct (mcp (prune (choices b))).    
  \end{lstlisting}
  That is, remove illegal configurations that arise from fixed entries.
  \(\Rightarrow 3^{40} = 12157665459056928801\) possibilities
\end{frame}

\begin{frame}{Third approach}
  Take one choice at a time:
  \begin{itemize}
    \item Find a free entry in the pruned board with minimum number of choices
    \item Generate new boards from these entries until only single entries are left
  \end{itemize}
  
  Notes:
  \begin{itemize}
    \item Very high order
    \item General recursion required
    \item Solves a Sudoku in a couple of seconds
  \end{itemize}
\end{frame}

\begin{frame}{Reasoning in Coq}
  Reasoning about correctness:
  \begin{itemize}
    \item If there is a solution, the solver will find it
    \item The solver will find all solutions
    \item Equivalence of all three solvers
  \end{itemize}
  
  Problems:
  \begin{itemize}
    \item Board type $[[CellVal]]$ not specific enough
    \item No meaningful induction over Sudoku boards
    \item Nested \texttt{map}s are hard to decompose
    \item The non-trivial solver is incredibly complex
    \item Couldn't even prove trivial properties
    \item Correctness is trivially proven in two cases and impossible in the third
  \end{itemize}
\end{frame}

\begin{frame}{Reasoning in Agda}
  Our new approach:
  \begin{itemize}
    \item Focus on trivial versions
    \item Goal: show equivalence of the two
    \item New board type is $Vec$ ($Vec$ $CellVal$ $boardsize$) $boardsize$
    \item Programming with dependent types in Coq is \emph{hard}...
    \item ... so we wrote the new version in Agda
  \end{itemize}
  
  Main goal: show that 
  
  $filter$ $correct$ $\cdot$ $mcp$ = $filter$ $correct$ $\cdot$ $mcp$ $\cdot$ $prune$
  
  \begin{itemize}
    \item Unfortunately, proving properties in Agda is hard as well
    \item but we're hopeful we can at least prove the important steps of the proof
  \end{itemize}
\end{frame}

\begin{frame}
  \frametitle{Thank you for your attention!}
  
  Any questions?
\end{frame}


%% notes
% - restrict data types
% - tried with vectors
% - proving simple stuff


\end{document}

\documentclass[a4paper,11pt]{article}
\usepackage[utf8]{inputenc}
\usepackage[T1]{fontenc}

\headsep1cm
\parindent0cm
\usepackage{amssymb, amstext, amsmath}
\usepackage{fancyhdr}
\usepackage{lastpage}
\usepackage{graphicx}
\usepackage{listings}

\lhead{\textbf{A Simple Sudoku Solver}}
\rhead{(Submission: 05.07.2013)}

\cfoot{}
\lfoot{Marco Eilers - F121763, Robert Schmidtke - F121550}
\rfoot{\thepage\ of \pageref{LastPage}}
\pagestyle{fancy}
\renewcommand{\footrulewidth}{0.4pt}

\setlength{\parskip}{4pt}

\begin{document}

\title{Dependently Typed Programming\\Verification Challenge: A Simple Sudoku Solver}
\author{Marco Eilers - F121763, Robert Schmidtke - F121550}

\maketitle
\newpage

% The final report about your verification challenge is due July 5th. Your report include the following elements:

% a brief description of the problem you worked on;
% a description of how you implemented the functions from your paper in Coq. What was hard? Is all the recursion structural? Or were there missing case branches? What design decisions did you make at this point;
% a description of the specification or properties that these functions satisfy. How far did you get with all the proofs? Were there any properties that were especially hard/easy to prove? Did you require any auxiliary lemmas?
% a conclusion, reflecting on your work. What would be needed to finish this project? What would you do differently next time?
% Basically, your report should provide documentation for your source files. I should be able to read your report, and then look at your source code without having to wonder 'what does is function/lemma/property for?' You can assume I am somewhat familiar with the pearls, but it's probably best to explain too much rather than too little. There is no fixed page length, but I'd expect you need more than two pages, but less than twenty.

\end{document}

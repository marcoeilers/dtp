\documentclass[a4paper,11pt]{article}
\usepackage[utf8]{inputenc}
\usepackage[T1]{fontenc}

\headsep1cm
\parindent0cm
\usepackage{amssymb, amstext, amsmath}
\usepackage{fancyhdr}
\usepackage{lastpage}
\usepackage{graphicx}
\usepackage{listings}
\usepackage{todonotes}

\lhead{\textbf{A Program to Solve Sudoku}}
\rhead{(Submission: 08.07.2013)}

\cfoot{}
\lfoot{Marco Eilers - F121763, Robert Schmidtke - F121550}
\rfoot{\thepage\ of \pageref{LastPage}}
\pagestyle{fancy}
\renewcommand{\footrulewidth}{0.4pt}

\setlength{\parskip}{4pt}

\begin{document}

\title{Dependently Typed Programming\\Verification Challenge: A Program to Solve Sudoku}
\author{Marco Eilers - F121763 \and Robert Schmidtke - F121550}

\maketitle

\abstract{This report documents the Coq implementation and verification of a Sudoku solver as described by Richard Bird's functional pearl \emph{A program to solve Sudoku}. The implementation follows closely the Haskell definitions described in the functional pearl. Problems arising from Coq's strictness when using advanced techniques such as general recursion and dependent types are discussed and sometimes alternative approaches in Agda are presented where Agda was easier to use. Two naive implementations and an advanced one are presented, the latter being able to solve a Sudoku puzzle in Coq in a matter of seconds. Verification is mostly restricted to the first two versions because of the complexity of the third approach. }

\newpage

% Basically, your report should provide documentation for your source files. I should be able to read your report, and then look at your source code without having to wonder 'what does is function/lemma/property for?' You can assume I am somewhat familiar with the pearls, but it's probably best to explain too much rather than too little. There is no fixed page length, but I'd expect you need more than two pages, but less than twenty.

\section{Problem Description}
\label{sec:prob-desc}
% a brief description of the problem you worked on;
The functional pearl describes multiple ways (with increasing efficiency) to solve the famous Sudoku puzzle automatically. In this report it is assumed that the reader is familiar with the Sudoku puzzle. The pearl aims at providing a solver that works on any \textit{box size} \(N\) such that the size of the Sudoku board is \(N^2 \times N^2\). Hence, even though there are types of Sudoku puzzles that are rectangular but not quadratic, the solver presented in the functional pearl is restricted to work on the quadratic ones only.

In addition to the abstraction over the box size \(N\) the solver does not require the values to be filled in (the \textit{cell values}) to be numbers but only a set of enough distinct elements to solve the puzzle. Also, the symbol to describe blank cells (the ones that are yet to be filled in) can be chosen arbitrarily. While the implementation presented in section~\ref{sec:impl} respects these abstractions, \(N = 3\), the numbers 1 to 9 for the cell values and '.' for blank cells have been chosen.

There are multiple ways to design an automatic Sudoku solver, the easiest to implement and most naive one being the enumeration of all board configurations and filtering out the invalid ones. This is done by assigning a list of possible cell values to each blank cell in the initial configuration and then generating the \textit{matrix Cartesian product} (MCP) of this board that results in every possible combination of cell values respecting the initial configuration of the puzzle. This approach results in a large number of generated boards; in fact it is claimed that around \(9^{40}\) possible combinations exist for a \(9 \times 9\) board with about half the cells filled initially (which is more than usual).

While certainly delivering correct results, this approach can be improved to generate a lot fewer board combinations which need to be checked for correctness: Instead of generating the MCP using all cell values for each blank cell, one can limit the number of generated boards by only considering cell values for blank cells that do not already violate the Sudoku rules. This means that technically possible choices that do not respect the "duplicate-constraint" in a row, column or \(N \times N\)-box are removed (\textit{pruned}), giving only \(3^{40}\) generated boards to check for correctness.

Since this number is still too large by far to be computed and checked in a reasonable amount of time, a third and completely different approach is considered. Instead of generating all possible board combinations at once, only one blank cell is considered at a time. The idea is that by looking at a blank cell in the puzzle with a minimum number of remaining valid choices for it to be filled and then generating new boards with this cell value fixed (e.g. at most 9 for an empty initial board) results in a smaller overall amount of generated boards. The process of generating new partially solved Sudoku puzzles by fixing a cell with fewest choices is called \textit{expansion}.

In the next section each of these three approaches' implementation in Coq will be described in detail.

\section{Implementation in Coq}
\label{sec:impl}
% a description of how you implemented the functions from your paper in Coq. What was hard? Is all the recursion structural? Or were there missing case branches? What design decisions did you make at this point;
The working implementation resides in the files \texttt{Sudoku.v} and \texttt{Misc.v}, the latter one containing auxiliary functions on \texttt{List}s and \texttt{Nat}s and some notations. There are also experimental implementations using \texttt{Vector}s in Coq and in Agda as well to explore the possibilities of constrained types (see section~\ref{sec:reas} for a more in-depth explanation).

\subsection{Misc.v}
Most of the functions contained here are trivial and do not require explanation that exceeds the comments in the source code itself. However some of them need a little bit of elaboration.
\begin{itemize}
  \item \texttt{combine\_prepend}: Since there is no \texttt{zipWith} equivalent in Coq that facilitates prepending each element of one list to each list in another list this helper function has been implemented. It takes a list of elements and a list of lists and prepends the \(i\)th element of the first list to the \(i\)th list of the second list.
  \item \texttt{cp}: This function takes a list of lists and produces a list of all possible pairs of the elements in these two lists (the Cartesian product). E.g. \texttt{cp [[1, 2], [3, 4]] = [[1, 3], [1, 4], [2, 3], [2, 4]]}.
  \item \texttt{mcp}: This function takes \texttt{cp} a step further by applying the Cartesian product to a list of lists of lists. This is done by first computing all possible pairs of nested lists and then computing the Cartesian product on the result. Note that depending on the size of the lists this may be a very expensive computation.
\end{itemize}

\subsection{Sudoku.v}
The implementation starts out with defining constants that have been discussed in section~\ref{sec:prob-desc}, fixing the solver to work with the most common variant of Sudoku: a \(9 \times 9\) puzzle with cell values from 1 to 9. Note that \texttt{cellval} has been chosen to be declared as a type rather than natural numbers to restrict the range of possible values such that reasoning about the structure later on does not become unnecessarily complicated. Custom tests for equality and blankness are provided as well. The \texttt{board} is then defined as a list of \texttt{row}s and a few (partially invalid) example boards are given for testing basic operations on boards.

Throughout the solver utilities are used for destructing the board into its logical parts: \texttt{rows} which just returns the board per definition; \texttt{cols} which transforms a list of \texttt{row}s into a list of \texttt{col}s (effectively computing the transpose) and \texttt{boxes} which gives the \(N^2\) regions of a Sudoku puzzle. These functions are not restricted to operate on \texttt{cellval}s but rather on any list of lists. This is due to the fact that they have to be used later on with more general functions such that a restriction here is not possible.

For the sake of simplicity \texttt{matrix\_choices} are defined as a list of lists of lists of \texttt{cellval}s. This simply describes a partially solved puzzle where instead of a fixed value for each cell there is a list of possible choices. \texttt{choose} is used while transforming a regular \texttt{board} to \texttt{matrix\_choices} (using \texttt{choices}) by either putting an already fixed cell value in a single element list, or assigning all possible cell values to an empty cell.

The naive approach then computes all possible boards using \texttt{mcp} and \texttt{filter}s the \texttt{correct} ones, where a board is considered correct if there are no duplicates in each row, column and box. Note that the correct criterion could be expanded, e.g. by requiring that all \texttt{cellvals} are used (and not more) and that no blank cells are included. However, by equivalence and design of the algorithm this is not necessary (since all blanks are replaced by all possible cell values). An example call is given on a board that only has very few blanks since this algorithm brings Coq to a grinding halt when run on a regular board (e.g. \texttt{solvable\_board}).

The second approach improves on the first one as described before: the only difference is that the generated choices considered during the MCP computation are pruned first. This is done by \texttt{delete}'ing all values from possible cell values for blank cells that are already \texttt{fixed} (i.e. are singleton lists and thus the only remaining choice for that cell) in the surrounding region (row, column or box). It is notable that this relatively simple adaption introduces quite some complexity to the solver. The example call for this approach is performed on the same simple board for the same reasons as before.

The remainder of the file is devoted to the third approach. It is best to look at the implementation top-down, i.e. consider \texttt{sudoku} first and then look at the ingredients by scrolling up again. Since the generated solution is a (usually a single-element) list of boards of single cell value list it is necessary to extract these using nested \texttt{map hd} applications. Note that \texttt{Blank} is only given as a type indicator for \texttt{hd} and should never be used since the application is on completely solved boards without blank cells.

\texttt{search} performs the actual work of this approach to solving Sudokus since it identifies a cell with fewest number of possible choices and \texttt{expand}s on it. Each expansion is pruned first to increase the number of fixed cells that will not be included in the next expansion. Since structural recursion on the \texttt{matrix\_choices} is impossible in this case (the recursive call is in the lambda expression for \texttt{map}) and well founded relations are very hard to define an artificial natural number argument has been introduced on which structural recursion is performed.

While this enabled actual implementation of the function it is certainly not the cleanest and most intuitive solution to this problem. Furthermore it is very hard to estimate how large the initial value for this number needs to be to allow for a sufficient level of recursion to solve the Sudoku completely. However, for the given \texttt{solvable\_board} a value of 6 was sufficient. This means that no search tree had greater depth before the sub-solution was either thrown away because of infeasibility or a solution had been found and pruning indeed has a great impact on the number of choices to be considered.

One alternative to the artificial natural number parameter could be to consider the overall number of remaining choices for a Sudoku as a measure for general recursion using Coq's Program Framework. This has been attempted in a second version of \texttt{search} and \texttt{sudoku} in Coq, but since recursion is not performed on a sub-term the resulting obligation is hard to prove.

Inside \texttt{search} there are three possible sub-branches (in the non-impossible branch):
\begin{enumerate}
  \item The given choices are blocked: there are duplicates or blank cells in the solution, which is thus invalid.
  \item There are only single element lists of cell values in the board: there is a solution and it can be returned as such.
  \item The board is not solved yet: generate more choices by expanding on the cell with fewest possible choices and recurse on each pruned choice.
\end{enumerate}

The second major ingredient is \texttt{expand}. First the number of fewest choices in the current configuration is determined and later used to identify a cell that is expanded upon. This is done by breaking the board in two parts, the first row of the second one containing the desired cell. Next, this row is broken into two again, this time breaking at the cell with fewest number of choices. Based on this splitting new overall boards are generated, each one with a different fixed value for that cell.

\texttt{best} is used to assess that a given list of possible choices is indeed the one with fewest (\(m\)) entries. \texttt{minchoice} returns the length of the shortest list of possible choices (but greater than 1, since that would correspond to a fixed cell) to determine the current minimum number of choices for all cells.

Despite being quite complex, \texttt{sudoku} is able to solve a regular board in a matter of seconds by not exploring invalid solutions too much.

\subsection{SudokuVector.v}
An incomplete implementation of the naive solvers which uses \texttt{Vector}s can be found in the file \texttt{SudokuVector.v}. It contains no executable Sudoku solver and is mainly included for completeness. The reasons for starting and abandoning this implementation will be explained below.

\subsection{Implementation in Agda}
The file \texttt{Sudoku.agda} contains a full implementation of the two naive Sudoku solvers using vectors as well as a mostly complete proof of their equivalence. Please note that loading this file in Emacs may take a long time (or at least does so on our computers) for reasons we do not quite understand.

It starts by defining some utility functions which are used in the later proofs, \texttt{cong}, \texttt{trans} and \texttt{sym}, as well as some functions for working with vectors.

The following definitions are similar to those found in the Coq version, except that vectors are used whereever possible instead of lists. Function and type names are the same as or very similar to the Coq version modulo differences in naming conventions (e.g. starting type names with capital letters in Agda where Coq uses small letters). 

As in the Coq implementation, \texttt{sudokuNaive} and \texttt{sudokuNaive2} are the functions corresponding to the first two versions of the solver.

The equivalence proof following these definition will be discussed below.

\section{Reasoning about the Solver}
\label{sec:reas}
% a description of the specification or properties that these functions satisfy. How far did you get with all the proofs? Were there any properties that were especially hard/easy to prove? Did you require any auxiliary lemmas?


\subsection{Specifications and Properties}
In his paper, Bird proves the equivalence of the first and second versions, and gives a very brief sketch for a proof of the equivalence of the first and third version. In these proofs, he uses several other facts about the functions he uses, which we will discuss now.
 
The extraction operations on boards \texttt{rows}, \texttt{cols} and \texttt{boxes} are involutive, meaning that \(f \cdot f = id\) for \(f \in \{rows,cols,boxes\}\). For \texttt{rows} this holds trivially since the board is defined as a list of rows and thus can be returned as-is. For \texttt{cols} and \texttt{boxes} (which uses \texttt{cols}) this is more involved. 

Bird gives several other properties that must hold for the functions used in the two naive versions:
\begin{align}
  &filter~(p \cdot f) = map~f \cdot filter~p \cdot map~f\\
  &map~f \cdot filter~p = filter (p \cdot f ) \cdot map~f\\
  &filter~(all~p) \cdot cp = cp \cdot map~(filter~p)\\
  &map~f \cdot mcp = mcp \cdot f~~\text{for \(f \in \{rows,cols,boxes\}\)}\\
  &filter~nodups \cdot cp = filter~nodups \cdot cp \cdot reduce
\end{align}

The first two laws are valid if \(f \cdot f = id\), in particular if \(f\) is either \(rows\), \(cols\) or \(boxes\).
The third law states that if lists from a Cartesian product are wanted for which all elements satisfy \(p\), they can be obtained by computing the Cartesian product on component lists whose elements satisfy \(p\).
The fifth law is the crucial property of \(reduce\): filtering lists from the Cartesian product which do not contain duplicates yields the same result as removing already fixed entries in a region from the possible choices for a cell value. 

Bird simply assumes that the above statements hold without proving any of them. One goal of our verification could therefore be to prove these five assumptions as well as the involution property of the extraction operations.

For the last approach no further specifications are given for the functions that are being used. 

In addition to that, it would be reasonable to show that all three versions are both sound and complete, meaning that
\begin{enumerate}
  \item for all input boards, the resulting list contains only valid solutions of the input board, and
  \item for all input boards, he resulting list contains \emph{all} valid solutions (if any) of the input board.
\end{enumerate}

For the naive approach, this is very simple: Since all possible configurations are generated and only the correct ones are filtered, it is guaranteed that the naive approach finds only possible solutions and does not discard valid solutions. The second approach is very similar but has one important difference: intermediate solutions are pruned to eliminate invalid configurations before filtering for correct results. So for the second approach to be sound and complete, it would suffice to show that it is equivalent to the first one. For this to be the case, the following must hold:
\begin{align*}
  filter~correct \cdot mcp \cdot choices &= filter~correct \cdot mcp \cdot prune \cdot choices\\
  \Rightarrow filter~correct \cdot mcp &= filter~correct \cdot mcp \cdot prune
\end{align*}

This says that pruning generated choices of the board produces the same overall solutions because only choices are pruned that would be discarded by the correctness filter anyway. Bird proves this in his paper using the facts stated above.


Since the third and more sophisticated implementation does not rely on exhaustively generating configurations and filtering correct ones, the equivalence proof would be more complicated. Bird only gives an outline of what such a proof might look like. Instead of proving equivalence it would also be possible to do quantification of all valid boards (valid in the sense that only legal symbols are used) and do induction over those in order to prove soundness. This approach has, of course, its own problems, which we will discuss in the next section.


\subsection{Verification}
Proving the soundness of the two naive approaches is trivial. Since incorrect solutions are filtered out as the last step in both of them, the resulting list can only contain correct ones. The two (very short) formal proofs of this in Coq can be  at the end of \texttt{Sudoku.v}.

While this was simple, soundness for the efficient implementation, completeness for any implementation but the first, or in fact anything else was almost impossible. In particular, we encountered the following problems:
\begin{enumerate}
  \item Representing Sudoku boards as lists of lists of cell values is not specific enough. Many of our functions expect inputs of size \(N^2 \times N^2\). For these functions, the properties we want to prove simply do not hold for lists of lists of any length. Even the most basic functions, like \texttt{rows} and \texttt{cols}, do not behave the way we need them to if list lengths are not restricted.
  \item There is no meaningful way to perform induction over Sudoku boards. If rows or columns are added or removed arbitrarily, the resulting matrix ceases to be a valid Sudoku board. While it is possible to perform induction over the box size $N$ of Sudoku boards, this does not actually help with proving anything.
  \item Generally, it can be hard to prove properties of functions dealing with nested lists. This is especially the case if a function's return value depends on values from different levels of the list hierarchy. We will see an example of this later.
  \item The third implementation is much more complex than the two naive ones, and works quite differently internally. Since Bird does not prove its correctness in any detail in the paper, we also have very few specifications for the functions it uses. These functions destruct, process and reassemble solutions in various non-trivial ways which do not allow us to check their correctness directly, which is why we would have to resort to proving the soundness or completeness of the solver as a whole.
  \item In addition to that, the third approach uses \texttt{search}, which, as discussed before, contains both an unreachable branch and an additional numerical parameter to allow for general recursion. This, as well as any other way of dealing with those two problems, makes it much harder to prove any properties about it.
\end{enumerate}

As a result of these problems, we decided to restrict our proofs in several ways:
\begin{itemize}
  \item We only reason about the two naive solutions. More specifically, since these have already been proven to be correct, we decided to prove the equivalence of the two solutions, following Bird's proof in the paper. Since the first version is trivially complete (since it generates all possible solutions and throws out the incorrect ones), this would show that the second version is complete as well.
  \item Instead of allowing arbitrary Ascii characters for cell values, a dedicated type has been introduced. This was done hoping the restriction would come in handy later on but has not been found particularly useful after all since reasoning about individual cells was not performed.
  \item We use vectors of fixed length instead of a loose list-of-lists representation for boards. As discussed above, \texttt{SudokuVector.v} contains an implementation of the two naive solvers. This implementation is, however, incomplete, because it was impossible to write \texttt{boxes} in Coq, which is central to the program. More specifically, we did not succeed in writing the function \texttt{groupBy} in a way that Coq's type checker would accept. Since Agda is known to be better at handling dependent types, we implemented both versions with vectors in Agda (successfully). 
\end{itemize} 

An alternative approach to using \texttt{Vector}s to restrict input lists to have a well defined length is Coq's Program Framework. It allows defining subtypes of lists to have a certain length, given implicit arguments, e.g.:\\
  \texttt{Program Definition listn \{n : nat\} \{A : Type\} :=\\
    \{ l : list A | length l = n \}.}\\
  Implicitly these subtypes are regular types in Coq, but whenever a conversion of a regular type to a subtype is performed (e.g. when a function's return type is a subtype of list having a certain length) a proof obligation is generated that this conversion is legal. However, when using these subtypes even just implementing the functions operating on them becomes very hard as matching on the implicit arguments is not easily done and many obligations to prove conversions to and from subtypes are generated. While being a powerful tool, using subtypes has been considered too complicated both for implementations and proofs.

While trying to prove properties of the new implementations, we soon discovered that Coq's ability to reason about dependently typed values is quite limited. When performing induction on a board (i.e. a vector of vectors of cell values, with both vectors having length \texttt{boardsize}), for example, Coq generated a case for an empty vector, although \texttt{boardsize} was defined as 9, and the vector therefore could not be empty. After unfolding the definition of \texttt{boardsize}, however, Coq completely refused to destruct the board. This and some other problems convinced us that it would be easier to use Agda for our main proof.

\subsubsection{Proving Bird's assumptions}
In order to reproduce Bird's equivalence proof, we first needed to prove his assumptions about the functions used in the Sudoku solvers. We were successful to a certain degree, but could not prove all of them. For the proofs of (1) and (2), see \texttt{filterMapFilterMap} and \texttt{mapFilterFilterMap} in \texttt{Sudoku.agda}. We could also prove the involution property of the extraction operations (see \texttt{rowsId}, \texttt{colsId} and \texttt{boxsId}), though the proofs for the latter are a bit of a hack, and we managed to prove (4) for \texttt{rows} only (see \texttt{rowsMapMcp}). We did not manage to prove (3) and (5), as well as (4) for \texttt{cols} and \texttt{boxs}. The main problem in all these cases were the functions \texttt{cp} (and \texttt{mcp}, which uses \texttt{cp}). The function looks like this:
\lstset{
  literate={λ}{{$\lambda$}}1
}
\begin{lstlisting}
cp : {a : Set} {n : Nat} -> Vec (L.List a) n 
       -> L.List (Vec a n)
cp [] = L._::_ [] L.[]
cp (xs :: xss) = 
  L.concat (L.map (λ x -> L.map (λ ys -> x :: ys) 
                  (cp xss)) xs)
\end{lstlisting}

It is easy to see why performing inductive proofs involving this function is not a promising strategy, since it has to evaluate its argument completely before calculating even parts of the return value. The properties mentioned above are therefore left unproved.

The only way to prove the involution property, even with vectors, was to simply decompose the input entirely. This enables Agda to execute the function entirely, so that the proof itself could simply use reflexivity. This is not elegant, and the proof needs to be adapted whenever the board size is changed (though the structure remains the same, of course), but more sophisticated approaches simply did not work out.

\subsubsection{Proving equivalence}
We first set out to prove that 
\begin{align*}
  filter~(allVec~nodups \cdot f) \cdot mcp \\= filter (allVec~nodups\cdot f) (mcp \cdot pruneBy~f)
\end{align*} 
for $f \in \{rows,~cols,~boxs\}$, following Bird's proof (p. 675 in the paper) quite closely. Using the lemmas mentioned above, we were able to prove this without any gaps. The proofs are named \texttt{step1} to \texttt{step9} and can also be found in \texttt{Sudoku.agda}. A remarkable feature of most of these proofs is that they look quite complicated and often require at least one additional lemma, although single steps often consist mostly of simple rewrites. This is due to Agda's rather complicated way of constructing proofs and would probably have looked a lot nicer when done in Coq.

Using those steps, we could construct proofs showing the abovementioned lemma for \texttt{rows}, \texttt{cols} and \texttt{boxs} (see for example \texttt{pruneCorrectRows}). We then used these to show that 
\begin{align*}
  filter~correct \cdot mcp = filter~correct \cdot mcp \cdot prune,
\end{align*}
mostly by rearranging proof terms and rewriting with theorems we proved before; again following Bird's proof (p. 676 in the paper). This, too, was successful and could be done without any gaps. 

We did, however, have problems proving some properties of \texttt{filter} in Agda, namely the lemmas \texttt{filterConjunction} and \texttt{filterSwap}. We proved these lemmas in Coq; the proofs can be found in the file \texttt{SudokuAgdaHelp.v}. They are left without proof in the Agda file. 

This and the relative ugliness of almost all out proofs in Agda raises the question if it would have been a better idea to prove at least some properties in Coq. Its support for performing varying degrees of simplification at will as well as much simpler rewrite mechanisms suggest that we could have saved a lot of time this way, and would have ended up with much more readable proofs.


% most basic functions are already ridiculously complex: cols, boxes
% no meaningful induction over boards
% restriction on basic functions
% operate on general lists
% restriction to cellval as a type did not yield immediate advantage
% equivalence of first two approaches trivial, third one DOWNRIGHT IMPOSSIBLE
% krücken: hd Blank to make it type check; 1000 for recursion; need to generalize some functions since they're used in complex chains and on other types as well after all 
% subtypes did not work either
% step-by-step proofs from paper in agda

\section{Conclusions and Further Work}
\label{sec:conc}
% a conclusion, reflecting on your work. What would be needed to finish this project? What would you do differently next time?
There are multiple reasons to why implementing and verifying the Sudoku solver with either Coq or Agda is hard:
\begin{itemize}
  \item Properties and specifications of the overall solver are hard to break down for individual component functions.
  \item Advanced general recursion is not easily implemented and it is not even sure if there exists a better implementation approach to \texttt{search}. Introducing dependent types (e.g. \texttt{Vector}s instead of \texttt{List}s) complicates this matter even further.
  \item As the non-trivial solver is of very high order reasoning about arguments to low-level functions with respect to the argument passed in to the solver is not straightforward.
\end{itemize}

The problem of solving Sudoku puzzles automatically seems to be a nice functional pearl and the Haskell implementations are elegant and short. Given the nature of the Coq framework it turns out that Coq is not ideal for general functional programming (e.g. requirements of functions to be total and quite restricted recursion). It is because of this that the Coq implementation sometimes lacks functional elegance: general recursion had to be defined introducing an artificial parameter on which structural recursion is performed (which in turn made reasoning about the recursion not logical but purely structural); impossible branches could not be eliminated by constraining input arguments; additional arguments (although not sensible) had to be provided for successful type checking (e.g. \texttt{hd Blank} in the non-trivial \texttt{sudoku} function). On the other hand, reasoning and proving properties of basic functions is easy and well supported.

Since functional programming in Agda is easier because of the better support of dependent types, implementing the naive solvers using constrained data types has been less of an effort. Unfortunately Agda's theorem proving abilities are less well developed, which is why proofs often have to be broken up into several pieces and are generally very verbose, sometimes to the point of being barely readable. 

As a result of the complexity of the problem, the strictness of the Coq framework and involved proving techniques in Agda, the solution presented resembles a mixed approach cherry-picking from the advantages of theoretical reasoning in the paper, Coq's interactive and straightforward proving mechanisms and Agda's support of dependent types and functional programming. While some Lemmas were straightforward to prove in Coq, it was not easily (or not at all) possible in Agda and vice versa. That is why our main equivalence proof contains both statements proven in Coq and Agda, and sometimes has to incorporate unproved claims from the paper in order to form a reasoning about the solver that is as complete as possible. Because of the functional programming restrictions in Coq and involved proving in Agda it was not feasible to construct an isolated complete proof of advanced properties in either Coq or Agda.

What has been discussed rather informally so far can be stated as clear facts of what needs to be done to finish the project and alternative solution approaches if the project was to be repeated:
\begin{itemize}
  \item Use Coq more often to prove properties. Even if implementation of functions in Coq is harder at times than in Agda, it pays off when it comes to proving specifications as using Agda's proof system is complicated. Effort could have been reduced here. 
  \item Discover and think about effects of implicit assumptions on the overall system early: require Sudoku puzzles as \(9 \times 9\) matrices from the start rather than trying to be as generic as possible to simplify proofs. The authors of the functional pearl were not concerned with computer assisted proofs of their solution which is why they support general \(N^2 \times N^2\) Sudoku puzzles. Restricting the scope of the Coq implementation to fixed size Sudoku puzzles with as clear a specification as possible is a better way to approach the verification problem.
  \item Do not underestimate how difficult it is to reason about 2 or sometimes 3 levels of nested lists or vectors. The more general the underlying data structure is, the more general the functions applied to them have to be which in turn introduces many unnecessary and complicated cases when reasoning about the properties of these functions.
\end{itemize}

To summarize, in most cases it was not hard to transfer the Haskell definitions of the functions presented into Coq. However, general and advanced proofs are difficult to handle in one system only, which is why there is usually a claim from the original paper involved as well as partial proofs in Coq and Agda. Defining properties of the problem early in the process and rigorous adherence to them is a promising way of verifying the solver solely and extensively in Coq.

\end{document}
